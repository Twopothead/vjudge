\chapter{编者序}

\begin{quotation}
  \begin{flushright}
{\LARGE \textbf{D}}{ie Philosophen haben die Welt nur verschieden interpretiert,\\
es kommt aber darauf an, sie zu verändern.} \par
\textit{哲学家们只是用不同的方式解释世界,而问题在于改变世界。}\par
—— {\small 马克思《关于费尔巴哈的提纲》}
  \end{flushright}
\end{quotation}

深夜,幽幽的荧屏,杂乱的代码,诡异的错误,不知所措的我们死死地盯着屏幕,试图解决实验里那些莫名其妙的问题。
每当这时,我们都不禁抱怨:代码为什么有这么多错误,注释为什么这么混乱,指导书内容又为何如此鸡肋。
抱怨之余我们也幻想了很多假如,相信曾挣扎于操作系统实验的人都曾这么想过:假如代码注释更完善些,
假如实验的体系和计组一样成熟,假如...\par
我们最大的愿望,便是有一本指南针般的指导书。它可能会穿插设置一些小练习让我们更容易上手并带来满足感,它可能会针对性地让我们对某些细节进行深度思考,它可能会从实验角度为我们展现一个真实的操作系统,它可能会让我们着迷于操作系统实验,它优雅而美丽,它平易而近人。假如那样,我们或许能更专注于操作系统本身设计吧。

但我们不希望我们只在心里假如,我们想做些什么。一路走来,我们耗费了太多的时间在不必要的地方,
因此,我们不希望这样的经历延续下去,一年又一年,一届又一届,我们想做些改变。

这正是我们重新整理并改写指导书的初衷,而你将要看到的,就是我们为改善操作系统实验而作出的努力。
仅以此书献给我们所热爱的学弟学妹们,愿它能带给你一次愉快而奇妙的编写操作系统的体验。

学生的潜力是无穷的,但我们还只是操作系统实验改革的起点。本书还有很多不足之处与不完善之处,
但是我们可以向你许诺,不再是大量无用材料的堆砌,不再是混乱无逻辑的排版,除很少部分的转载外,
其他内容都是我们日日夜夜呕心沥血的原创。同时,秉着减轻代码难度,
加强思考深度的初心,我们创造了许多思考训练。所有的思考训练都是我们精心设计,是世上绝无仅有的。
所以我们请求你,希望你多花些时间来看看这本书,自主思考,勇于探索,并鼓励你与伙伴积极交流。

哪天我们若能在路上听到你和小伙伴神采飞扬地讨论操作系统实验中的某个细节,
那时或许我们会在心里欣慰地轻叹一声

\vspace*{0.5\baselineskip}

“为了这帮兔崽子,值了!”

%% TODO: 下面的部分需要完善,阐述整个指导书的一些约定,各部分内容的介绍。
在重编指导书的过程中,我们对于读者的知识掌握情况做了如下的假设。

\begin{itemize}
  \item 熟练掌握C语言,特别是指针、结构体的使用。
  \item 对于MIPS体系结构有基本的了解(正常完成了计算机组成课程实验)。
  \item 初步了解中断、异常等概念。
\end{itemize}

指导书有三种特殊标记的段落:Note、Exercise和Thinking。
\begin{description}
\item[Note] 表示该段内容是补充说明,或者课外延展,可以选择性地跳过。
\item[Exercise] 我们需要完成的实验任务,我们将在服务器端对代码是否合格进行检查。
\item[Thinking] 思考训练,无固定答案,但是需要将你的见解写在实验报告中。
注意这些思考训练不同于你以往所做的题目,它由\textbf{关键代码、见解、参考资料}
组成。你需要使用参考资料与你认为的关键代码对你的见解进行支撑,为了做到这一点你可能需要做很多功课。
这部分内容将作为操作系统实验成绩的重要参考。我们没有规定固定格式,只需要在说明前标明Thinking序号即可。
\end{description}


\begingroup
\vspace*{\baselineskip}

\noindent\centering\rule{0.9\textwidth}{0.2pt}\\[\baselineskip]

\noindent\LARGE \textbf{编者寄语} \small (按姓名汉语拼音顺序排列)

\begin{quotation}
  天道酬勤。
  \begin{flushright}
  ——何涛
  \end{flushright}
\end{quotation}

\begin{quotation}
行胜于言。
  \begin{flushright}
  ——刘乾
  \end{flushright}
\end{quotation}

\begin{quotation}
  一个人可以走得很快,但一群人可以走得更远,我愿与你们一路同行,分享彼此的经验与知识。
  \begin{flushright}
  ——王鹿鸣
  \end{flushright}
\end{quotation}

\endgroup
